\chapter{Introduction}

Web applications usually run on two sides that we call a server and a client.
The sides communicate with each other by Internet Protocols, where
\ac{HTTP} is the fundamental communication standard.
Users use web browsers for requesting the server. The server sends back a response, containing the desired data.
The data can represent a web page or an attachment like a file or raw data.
The browser is responsible for interpreting and rendering the web page described by \ac{HTML}.
The \ac{CSS} language accompanies HTML by enriching the web page with broad graphical content. 
\par
The server´s task is to process, to collect and to serve the data requested by the client.
The most popular language for server-side scripting is currently PHP. 
\par
The combination of CSS and HTML can be sufficient for creating a standard web page.
However, a modern web application needs to manipulate the web page structure, depending on user behavior, in a more sophisticated way than CSS and HTML currently offer.
This type of application needs to use the browser as an execution environment. The environment should be able to change the web page structure, to react to the events, to save an application state, and to control the browser behavior. 
The scripting language called JavaScript became a browser standard for writing a client-side code inside most browsers as for example in Google Chrome, Safari, Opera, and Mozzila.
\par
Although JavaScript is a powerful language, it is not appropriate for all scenarios and users.
The reason can be dynamic typing or just a user practice with other languages.
Despite the urge to write a client-side code in a different language, many technologies like Silverlight, which runs C\# code in a browser, or Adobe Flash Player with Actionscript were deprecated due to insufficient support across the browsers.
\ac{WASM} \cite{online:wasmWiki} was developed to offer a portable binary-code format for executing programs inside a browser in 2015.
WASM targets to enable secure and high-performance web applications.
The advantage of WebAssembly is that it is a compilation target for many programming languages.
WASM and JavaScript interoperate and utilize in a browser both of the language advantages.
Since December 2019, when \ac{W3C} has begun recommending WebAssembly, it is easy to migrate other languages to the browsers supporting this recommendation.
\par
Many projects can be compiled to the WASM.
For example, the project PHP in browser \cite{online:pib} enables running a PHP script inside our browser using predefined JavaScript API or standard HTML tag.
Another project is an open-source framework Blazor \cite{online:blazor} developed by Microsoft.
Blazor provides a runtime environment, libraries, and interoperability between JavaScript and C\# enabling creating dynamic web pages in C\#.
\par
The .NET and PHP popularity led to the creation of the Peachpie compiler \cite{online:peachpie}.
Peachpie compiles PHP to .NET and thus enables interoperability between the languages.
Peachpie is usually used to connect a frontend written in PHP with a backend written in C\#. This utilizes both aspects of the languages on the server side.
\par
Peachpie allows using PHP in Blazor.
Although Blazor can straightforwardly reference compiled PHP by Peachpie, the collaboration between the code and Blazor seems complicated.
Methods of how to utilize PHP scripts as a part of a Blazor website are not clear. 
This thesis focuses on identifying use cases that will make use of the integration opportunity between Peachpie and Blazor. The thesis also suggests a solution by creating a library \texttt{Peachpie.Blazor} to execute and render compiled PHP scripts in a browser.
Blazor is used as an execution environment for these scripts.
\texttt{Peachpie.Blazor} tries to achieve two goals.
The first goal is to implement the support for using compiled PHP scripts with Blazor because there is no existing library that supports the integration.
The second goal is to enable the web development on a client side with PHP.
\par
The integration between Peachpie and Blazor can yield the following benefits.
The community of PHP developers is significant.
Thus, many PHP libraries enable working with user's data, pdf, graphics and offer handy tools.
The possibility to migrate the PHP language together with its conventions to the browser will impact developing dynamic web applications thanks to the PHP community and its libraries.
It can join PHP and C\# developers to collaborate with their programming languages using a minimum knowledge of the integration.
Another interesting functionality of this idea is a full C\#, PHP, and JavaScript interoperability which offers more options for developers and future extensions.
\par
The first chapter is about analysis of the related work, alongside with descriptions of the technologies used in the integration.
The second chapter analyses running PHP on a client side and other problems related to used technologies.
The third gives detailed description of the library's functionality.
There are examples that demonstrate how to use all aspects of \texttt{Peachpie.Blazor} in chapter 4.
In chapter 5, we can see benchmarks that show the limits of the implementation.
The last chapter relates to a conclusion of this approach of executing PHP scripts in a browser.

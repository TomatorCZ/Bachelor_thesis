\chapter{Attachments}

\begin{table}[H]
\centering
\begin{tabular}{ | m{11em} | m{22em}| } 
\hline
Folder & Description \\
\hline
\texttt{bin/} & Executable files of examples \\
\texttt{src/Peachpie.Blazor/} & Source code of library implementing the proposed solution \\
\texttt{src/examples/} & Source code of mentioned examples \\
\texttt{src/benchmarks/} & Source code of mentioned benchmarks \\
\texttt{src/templates/} & Source code of mentioned templates \\
\texttt{data\_benchmarks/} & Measured data using the benchmarks \\
\texttt{docs/} & Autogenerated technical documentation of the library, user manual, and thesis \\
\texttt{nugets/} & The library and templates packed into nugets. \\
\hline
\end{tabular}
\end{table}

The attachments contain a user manual that helps to get familiar with the library API and describes how to get started by steps.

\section{Build and run}

The library, examples, and benchmarks are .NET solutions, which can be opened by \ac{VS} 2019.
The \texttt{bin} folder consists of compiled examples with supports for various OS and architectures like Windows (x86, x64), Unix, OSX, or Linux (x86, x64, arm).
The .NET 5.0 runtime, which is available from \url{https://dotnet.microsoft.com/download/dotnet/5.0}, is required to run binaries.
After the installation, the particular example can be run by moving to the \texttt{bin/NameOfExample} directory and run the command line below:
\par
\begin{code}[frame=none]
dotnet BlazorApp.Server.dll
\end{code}
\par
Then, the website can be accesed by navigation to \url{https://localhost:5001}, or \url{http://localhost:5000}.
\par
The .NET 5.0 SDK, which is available from the same website as the runtime, is required to build .NET solutions.
After the installation, the \texttt{Peachpie.Blazor} library has to be compiled and packed. 
This is done by moving to the \texttt{src/Peachpie.Blazor/Peachpie.Blazor} directory and run the commands below:
\par
\begin{code}[frame=none]
dotnet build --configuration Release
dotnet pack --configuration Release
\end{code}
\par
Then, an example can be compiled by moving to its directory containing the \texttt{BlazorApp.Server} project and run the command below using the NuGet generated by the previous step:
\par
\begin{code}[frame=none]
dotnet build --configuration Release --source PathToTheLibraryNuGet
\end{code}
\par
After the compilation, the server can be launched by the instructions mentioned at the beginning of this section.
\par
The template compilation can be done by running a PowerShell command \texttt{./build/build.ps1} in the template folder.
It results in a NuGet package, which can be added to VS template collection.
The user manual gives instructions on adding the templates to VS.
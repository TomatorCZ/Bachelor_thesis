\chapter{Instructions for getting started}

The library contains Blazor components and helper classes enabling navigation and execution of PHP scripts, defined in a Peachpie project, inside a browser. 
It is available as a NuGet package, which can be found in the attachment. 
There are two ways how to get started. We created templates, which can be adjusted to fit your goal. 
The second way creates a new Blazor project referencing PHP scripts. 
We recommend beginning with the first (easier) way.
\par
\begin{itemize}
\item [\textbf{Prerequisities}]
\item .NET 5.0 SDK - \url{https://dotnet.microsoft.com/download/dotnet/5.0}
\item Visual Studio 2019 (Optional) - \url{https://visualstudio.microsoft.com/vs/}
\item \textit{Peachpie.Blazor.1.0.0-alpha.nupkg} - A part of the attachment
\item \textit{Peachpie.Blazor.Templates.1.0.2.nupkg} - A part of the attachment (Required by the first approach of creating a website)
\end{itemize}
\par
\begin{itemize}
\item [\textbf{From templates}]
\item [Step 1 -] Install the templates by \texttt{dotnet new -i}
\item [Step 2 -] Add a new source, where \textit{Peachpie.Blazor.1.0.0-alpha.nupkg} can be find by the package manager, to the \textit{NuGet.Config} file. It is usually located in \textit{\%appdata\%/NuGet/NuGet.Config/} on Windows or \textit{~/.nuget/NuGet/NuGet.Config} on Mac.
\item [Step 3 -] Choose a template depending on your intention:
\begin{itemize}
\item [\textbf{Peachpie Blazor Web} -] A simple PHP website running in browser
\item [\textbf{Peachpie Blazor Hybrid} -] A simple Blazor website combining PHP and Razor Pages
\end{itemize}
\item [Step 4 -] Create the project by \texttt{dotnet new blazor-hybrid} or \texttt{dotnet new blazor-web}.
\item [Step 5 -] Modify the \texttt{BlazorApp.Client} or \texttt{PHPScripts} (Optional).
\item [Step 6 -] Navigate to \textit{BlazorApp/Server/} and run \texttt{dotnet run}.
\item [Step 7 -] Access \url{https://localhost:5001}.
\end{itemize}
\par
\begin{itemize}
\item [\textbf{From Blazor project}]
\item [Step 1 -] Create Blazor WebAssembly project with ASP.NET Hosting and targeting .NET 5.0.
\item [Step 2 -]  Install Peachpie Visual Studio extension (\url{https://www.peachpie.io/getstarted}).
\item [Step 3 -] Create a Peachpie Class library project.
\item [Step 4 -] Add a new source, where \textit{Peachpie.Blazor.1.0.0-alpha.nupkg} can be find by the package manager.
\item [Step 5 -] Add references to the library in \texttt{BlazorApp.Client}, \texttt{BlazorApp.Server} and Peachpie projects.
\item [Step 6 -] Add \texttt{<script src="\_content/Peachpie.Blazor/php.js"></script>} to head of \textit{index.html} located in \textit{BlazorApp.Client/wwwroot/} folder.
\item [Step 7 -] Add assemblies containing PHP script to \texttt{WebAssemblyHostBuilder} (located in \textit{BlazorApp.Client/Pragram.cs}) using the \texttt{AddPhp} extension method.
\item [Step 8 -] Modify projects (optional).
\item [Step 9 -] Launch the server by \texttt{dotnet run} in \textit{BlazorApp/Server/} folder. 
\item [Step 10 -]  Access \url{https://localhost:5001}.
\end{itemize}

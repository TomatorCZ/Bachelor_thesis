\chapter{Problem analysis}

This chapter describes the main problems together with using technologies to help to solve them.
In the end, we will introduce the proposed solution to the problem.

Current problems of migrating PHP to a browser divides into different types of tasks that have to be done.
In the beginning, we have to load scripts alongside the Blazor app into a browser. 
Afterward, navigation to the scripts is essential for making the PHP website client-side.
The ability to find the desired script should be adopted to Blazor's environment in order to combine PHP scripts with Razor pages.
Mocking the server role on the client-side will be another important feature to make this migration familiar with standard PHP usage.
It consists of managing superglobals like GET or POST, file management.
Because we can save the application state, we bring a new feature of choosing the preservation of the script's context to the next evaluation.
The interaction with a user is a challenge for PHP due to its server role.
And the last problem will be with evaluating the PHP code.
Rendering the whole page is a demanding computation.
It can be critical for sections, which tend to change their content often.
 
\subsection{Proposed solution}
The main idea of migrating PHP to a browser is to integrate Peachpie with Blazor.
In the beginning, we have to think of how we can put the scripts into a browser.
Peachpie can solve it.
It compiles our scripts into .NET assembly, which consists of all information about the scripts.
We can reference it from a Blazor app as a standard CSharp code when we have the assembly.
We have to ensure that compiler will think the application uses the assembly due to cutting unnecessarily assemblies mentioned in the Blazor section.
The process results in loading the assembly alongside the Blazor application into a browser.
However, this could be considered as the major part of the problem.
How to reference and evaluate the scripts is not clear, and there will be many decisions that will not be silver bullets.

We can use a Blazor component class in order to represent a particular script in the Blazor application.
This component should be able to find and evaluate a specified script from the assembly.
This approach can benefit from the component's reusability.
Afterward, we will be able to compose the component with others to make the desired layout.
Before that, we have to make finding and evaluation the script clear.

From this time, we have to distinguish between the purposes of the PHP code.
We will create two components for each of them.
We make the reason clear later in the section and describe detail in the following chapter.

The first purpose is to free the script from Blazor.
It is done by finding the script by the name obtained from a component's parameter or the URL.
The script's evaluation is done via caching its output and adding it as markup text to the RenderTreeBuilder.
It is is a good approach for transparent use of Blazor, but it is ineffective for often rendering.

\change[inline]{User interaction -> forms}
\change[inline]{Superglobals get, post, files}

The second purpose is to offer the complete interface of Blazor.
An inheritance can achieve this.
The PHP class can inherit the CSharp component class due to Peachpie.
The consequence of this is accessible Blazor functions for rendering the content.

However, these purposes are different. They can be combined due to a Component interface.

\change[inline]{Full control over the rendering}







\chapter{Problem analysis}

The chapter devides the problem of running PHP on the client-side into two parts.
The first part pays attention to adaptation and changing a PHP paradigm on the client-side.
The second part describes the integration of Blazor and Peachpie. 
The advantage of the application's client's part is easy preserving an application state.
This is achieved by a browser storage, which remains until the application is shutdown.
The imposibillity of preserving an application state on the server-side causes stateless HTTP protocol. There is an existing way called PHP sessions, but it has some disadvanteges.

The transfer of PHP to the client-side has several problems. 
The first of them is changing a DOM structure with respect to user's interaction with the page.
A standard way how to interact is forms.
The second of them is the server support.
Superglobals like GET was filled transparently.
GET variable can be obtained by an URL processing, but POST is a seperate information wrapped in request.
Another data contained in the request are files.







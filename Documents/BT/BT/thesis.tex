%%% The main file. It contains definitions of basic parameters and includes all other parts.

%% Settings for single-side (simplex) printing
% Margins: left 40mm, right 25mm, top and bottom 25mm
% (but beware, LaTeX adds 1in implicitly)
\documentclass[12pt,a4paper]{report}
\setlength\textwidth{145mm}
\setlength\textheight{247mm}
\setlength\oddsidemargin{15mm}
\setlength\evensidemargin{15mm}
\setlength\topmargin{0mm}
\setlength\headsep{0mm}
\setlength\headheight{0mm}
% \openright makes the following text appear on a right-hand page
\let\openright=\clearpage

%% Settings for two-sided (duplex) printing
% \documentclass[12pt,a4paper,twoside,openright]{report}
% \setlength\textwidth{145mm}
% \setlength\textheight{247mm}
% \setlength\oddsidemargin{14.2mm}
% \setlength\evensidemargin{0mm}
% \setlength\topmargin{0mm}
% \setlength\headsep{0mm}
% \setlength\headheight{0mm}
% \let\openright=\cleardoublepage

%% Generate PDF/A-2u
\usepackage[a-2u]{pdfx}

%% Character encoding: usually latin2, cp1250 or utf8:
\usepackage[utf8]{inputenc}

%% Prefer Latin Modern fonts
\usepackage{lmodern}

%% Further useful packages (included in most LaTeX distributions)
\usepackage{amsmath}        % extensions for typesetting of math
\usepackage{amsfonts}       % math fonts
\usepackage{amsthm}         % theorems, definitions, etc.
\usepackage{bbding}         % various symbols (squares, asterisks, scissors, ...)
\usepackage{bm}             % boldface symbols (\bm)
\usepackage{graphicx}       % embedding of pictures
\usepackage{fancyvrb}       % improved verbatim environment
\usepackage{natbib}         % citation style AUTHOR (YEAR), or AUTHOR [NUMBER]
\usepackage[nottoc]{tocbibind} % makes sure that bibliography and the lists
			    % of figures/tables are included in the table
			    % of contents
\usepackage{dcolumn}        % improved alignment of table columns
\usepackage{booktabs}       % improved horizontal lines in tables
\usepackage{paralist}       % improved enumerate and itemize
\usepackage{xcolor}         % typesetting in color
\usepackage{acronym}		% abbreviations

\usepackage{float}			% allows to position images mor preciously.
%%% Basic information on the thesis

% Thesis title in English (exactly as in the formal assignment)
\def\ThesisTitle{Client-side execution of PHP applications compiled to .NET}

% Author of the thesis
\def\ThesisAuthor{Tomáš Husák}

% Year when the thesis is submitted
\def\YearSubmitted{2021}

% Name of the department or institute, where the work was officially assigned
% (according to the Organizational Structure of MFF UK in English,
% or a full name of a department outside MFF)
\def\Department{Department of Software Engineering}

% Is it a department (katedra), or an institute (ústav)?
\def\DeptType{Department}

% Thesis supervisor: name, surname and titles
\def\Supervisor{RNDr. Filip Zavoral, Ph.D.}

% Supervisor's department (again according to Organizational structure of MFF)
\def\SupervisorsDepartment{Department of Software Engineering}

% Study programme and specialization
\def\StudyProgramme{Computer Science (B1801)}
\def\StudyBranch{ISDI (1801R049)}

% An optional dedication: you can thank whomever you wish (your supervisor,
% consultant, a person who lent the software, etc.)
\def\Dedication{%
Dedication.
}

% Abstract (recommended length around 80-200 words; this is not a copy of your thesis assignment!)
\def\Abstract{%
Peachpie is a modern compiler enabling the compilation of PHP scripts into .NET.
Blazor is a new part of the ASP.NET platform offering the usage of C\# on a client side due to a new web standard, WebAssembly.
Although there exists a project providing the execution of PHP scripts inside a browser using WebAssembly, this thesis explores a new approach of the execution based on the integration between Peachpie and Blazor.
This exploration aims at enabling PHP programmers to move the PHP execution to a client side with the advantages of the Blazor environment.
However, the difference of used technologies limits usage possibilities.
}

% 3 to 5 keywords (recommended), each enclosed in curly braces
\def\Keywords{%
{PHP} {.NET} {Blazor} {Peachpie}
}

%% The hyperref package for clickable links in PDF and also for storing
%% metadata to PDF (including the table of contents).
%% Most settings are pre-set by the pdfx package.
\hypersetup{unicode}
\hypersetup{breaklinks=true}

% Definitions of macros (see description inside)
\include{macros}

% Title page and various mandatory informational pages
\begin{document}
\include{title}

%%% A page with automatically generated table of contents of the bachelor thesis

\tableofcontents

%%% Each chapter is kept in a separate file
\chapter{Introduction}

Web applications usually run on two sides that we call a server and a client.
The sides communicate with each other by Internet Protocols, where
\ac{HTTP} is the fundamental communication standard.
Users use web browsers for requesting the server. The server sends back a response, containing the desired data.
The data can represent a web page or an attachment like a file or raw data.
The browser is responsible for interpreting and rendering the web page described by \ac{HTML}.
The \ac{CSS} language accompanies HTML by enriching the web page with broad graphical content. 
\par
The server´s task is to process, to collect and to serve the data requested by the client.
The most popular language for server-side scripting is currently PHP. 
\par
The combination of CSS and HTML can be sufficient for creating a standard web page.
However, a modern web application needs to manipulate the web page structure, depending on user behavior, in a more sophisticated way than CSS and HTML currently offer.
This type of application needs to use the browser as an execution environment. The environment should be able to change the web page structure, to react to the events, to save an application state, and to control the browser behavior. 
The scripting language called JavaScript became a browser standard for writing a client-side code inside most browsers as for example in Google Chrome, Safari, Opera, and Mozzila.
\par
Although JavaScript is a powerful language, it is not appropriate for all scenarios and users.
The reason can be dynamic typing or just a user practice with other languages.
Despite the urge to write a client-side code in a different language, many technologies like Silverlight, which runs C\# code in a browser, or Adobe Flash Player with Actionscript were deprecated due to insufficient support across the browsers.
\ac{WASM} \cite{online:wasmWiki} was developed to offer a portable binary-code format for executing programs inside a browser in 2015.
WASM targets to enable secure and high-performance web applications.
The advantage of WebAssembly is that it is a compilation target for many programming languages.
WASM and JavaScript interoperate and utilize in a browser both of the language advantages.
Since December 2019, when \ac{W3C} has begun recommending WebAssembly, it is easy to migrate other languages to the browsers supporting this recommendation.
\par
Many projects can be compiled to the WASM.
For example, the project PHP in browser \cite{online:pib} enables running a PHP script inside our browser using predefined JavaScript API or standard HTML tag.
Another project is an open-source framework Blazor \cite{online:blazor} developed by Microsoft.
Blazor provides a runtime environment, libraries, and interoperability between JavaScript and C\# enabling creating dynamic web pages in C\#.
\par
The .NET and PHP popularity led to the creation of the Peachpie compiler \cite{online:peachpie}.
Peachpie compiles PHP to .NET and thus enables interoperability between the languages.
Peachpie is usually used to connect a frontend written in PHP with a backend written in C\#. This utilizes both aspects of the languages on the server side.
\par
Peachpie allows using PHP in Blazor.
Although Blazor can straightforwardly reference compiled PHP by Peachpie, the collaboration between the code and Blazor seems complicated.
Methods of how to utilize PHP scripts as a part of a Blazor website are not clear. 
This thesis focuses on identifying use cases that will make use of the integration opportunity between Peachpie and Blazor. The thesis also suggests a solution by creating a library \texttt{Peachpie.Blazor} to execute and render compiled PHP scripts in a browser.
Blazor is used as an execution environment for these scripts.
\texttt{Peachpie.Blazor} tries to achieve two goals.
The first goal is to implement the support for using compiled PHP scripts with Blazor because there is no existing library that supports the integration.
The second goal is to enable the web development on a client side with PHP.
\par
The integration between Peachpie and Blazor can yield the following benefits.
The community of PHP developers is significant.
Thus, many PHP libraries enable working with user's data, pdf, graphics and offer handy tools.
The possibility to migrate the PHP language together with its conventions to the browser will impact developing dynamic web applications thanks to the PHP community and its libraries.
It can join PHP and C\# developers to collaborate with their programming languages using a minimum knowledge of the integration.
Another interesting functionality of this idea is a full C\#, PHP, and JavaScript interoperability which offers more options for developers and future extensions.
\par
The first chapter is about analysis of the related work, alongside with descriptions of the technologies used in the integration.
The second chapter analyses running PHP on a client side and other problems related to used technologies.
The third gives detailed description of the library's functionality.
There are examples that demonstrate how to use all aspects of \texttt{Peachpie.Blazor} in chapter 4.
In chapter 5, we can see benchmarks that show the limits of the implementation.
The last chapter relates to a conclusion of this approach of executing PHP scripts in a browser.

\chapter{Existing technologies}

In the beginning, I will introduce an existing solution of how to run PHP in a browser. 
The project \cite{Pib} aims to use compiled PHP interpreter into WebAssembly, which allows evaluating a PHP code.
The page has to import a specialized module php-wasm. 
A PHP code is evaluated by writing a specialized script block or manually by JavaScript and API.
PHP can afterward interact with JavaScript using a specialized API.
At first glance, that might be a good enough solution, but they are several parts that can be problematic due to PHP semantics.
The solution doesn't solve globals. 
This is reasonable because this is the server's job, but you are not able to get information about a query part or handling forms without writing a JavaScript code.
The next problem is navigating how a script can navigate to another script without an additional support code which has to be JavaScript.
These problems can be solved by following technologies and their integration.

\section{WebAssembly}

\cite{WebAssembly} is a new code format that can be run in today's browsers. 
It has a compact byte format, and its performance is near to a native code. 
WebAssembly is designed to be a compiling target of popular low-level languages like C or C++ due to its memory model. 
It should be able to support languages with carbage collector in the future. 
The advantage of this format is a similarity with Javascript modules ES2015 after compilation into a machine code. 
This enables browsers to execute it by a JavaScript runtime. 
So its security is as good as a code written in Javascript. 
Because of the same runtime, WebAssembly can call Javascript and vice versa.

\cite{Threads} support is currently discussed nowadays and appears to be realistic.
After all, new versions of Google Chrome experiments with proper multi-threading support despite the chance of vulnerability.
A replacement of multi-threading can be web workers mentioned \cite{WebWorkers} article.
The worker's limitation is communication with UI thread only by messages.

Despite supporting to run WebAssembly in a browser, the browser cannot load it as a standard ES2015 module yet.
WebAssembly JavaScript API was created in order to be able to load a WebAssembly to a browser using JavaScript.

\section{Mono}

Mono is a .NET runtime that aims to mobile platforms. 
Recently, they started to support \cite{compilation} into WebAssembly.
This support allows executing CIL inside browsers.
The compilation has two modes.
The first one is compilation Mono runtime with all using assemblies.
The second only compile Mono runtime, which then can execute .dll files without further compilation of them into WebAssembly.
A consequence of these compilations into WebAssembly is enabling to call Javascript and WebAPI from .NET.

\section{Blazor}

Blazor is a framework that provides a convenient way how to write dynamic web pages using CSharp.
Blazor platform is divided into two \cite{Hosting_models} which have different approaches to creating web applications. 
The first one is referred to as Blazor Server App and has a similar methodology to a standard website written in PHP.
An interesting innovation is SignalR which is a communication protocol between the server and a client.
However, this thesis uses the second model, which Microsoft refers to as Blazor WebAssembly App enabling offline support after loading the app into a browser.

\subsection{Blazor WebAssembly App}
From now on, I will use Blazor App to refer Blazor WebAssembly App.
Blazor App can be divided into two parts.
The first part serves the main WebAssembly application and its additional resources, which can be requested during runtime.
The second part is WebAssembly wrapped together with an additional user code.
The division enables to choose of a place for the implementation of business logic.
If there is a bad connection, we can move the majority of business logic to the client and use the server for connection to a database; otherwise, we can use the client only for rendering the page. It consists of the following components. 
Kestrel with ASP.NET libraries provides the server part of an application.
Mono runtime compiled to WebAssembly runs CSharp code inside a browser.
WebAssembly is essential for being able to interact with DOM and JavaScript using CSharp without an additional plugin, which was necessary for older technologies like Microsoft Silverlight.
Blazor's libraries provide constructs for manipulation with DOM and WebAPI together with rendering the page and JavaScript interop.
And there is a user's code that using the libraries for creating dynamic pages with CSharp.
A better imagination, how the app is situated on the client-side, can be represented by the figure \ref{img01:wasm} copied from \cite{Glick2018}.

\begin{figure}[H]\centering
\includegraphics[width=140mm, height=100mm]{./img/BlazorExecution}
\caption{Running a Blazor WebAssembly App on client-side.}
\label{img01:wasm}
\end{figure}

Technical details of interop with a browser are one part of the Blazor App.
The main part is the architecture of the libraries.
A common approach how to create a page is using the markup language Razor.
There already exists Razor in standard ASP.NET website where .cshtml extensions consist of this markup.
Unfortunately, the markup used in BlazorApp has the same name.
From now on, I will use Razor for the markup language, which is the content of .razor files in Blazor App.
Because interleaving HTML with other languages turns out to be helpful, the Razor uses special characters to identify CSharp code in HTML and convert it to rich content pages.
A significant purpose of Razor is for generating CSharp structures, which represent parts of a page, during compilation time.
These structures have a complex interface for rendering a page, so the markup is there in order to free users using a complicated mechanism for putting a page together.

Blazor introduces a Component that can represent a whole page or the part of it.
Components can be arbitrarily put together in order to form the desired page.
They can have different purposes. For example, a Router takes care of routing the right page whenever the navigation is triggered.
Alongside components, a dispatcher supplies additional services like logger to components when they are creating.
The last item, which is not used transparently, is a WebAssemblyHost builder.
The builder configures the application and prepares the renderer used by components to render their content.

Balzor presents its own virtual DOM to reduce changing a DOM directly in a browser to its demanding performance.
A component works with RenderTreeBuilder, which provides an interface for adding content to the virtual DOM.
The usage of RenderTreeBuilder is complex due to Blazor's diff algorithm, which is used afterward.
RenderTreeBuilder is just a superstructure over Renderer, which is responsible for updating the page.
The diff algorithm is used to minimize the browser's DOM  update after all components used RenderTreeBuilder to render their content.
This algorithm used sequence numbers for parts of HTML to identify modified sections.
Sequence numbers respond to an order of RenderTreeBuilder's instructions in the source code.
A benefit of this information is detecting loops and conditional statements to generating smaller updates of DOM.  
It follows the browser's DOM update, which is executed by Blazor's JavaScript support code called through Mono runtime.

The process of bootstrapping the Blazor App to a browser follows these steps. 
Kestrel gets a request for a page that is contained in Blazor App. 
The server responses with the index.html page, which contains references to JavaScript support code (This code is referred to as blazor.js and mono.js in the figure \ref{img01:wasm}) responsible for loading and running the runtime with the application part.
The runtime runs the application using the Main method in Blazor App.
The remaining interactions are maintained by event handling.
I distinguish two types of events.
The first type is navigation.
The \cite{navigation} can be triggered by an anchor, form, or filling up the URL bar.
The URL bar is handled separated by a browser.
JavaScript can influence the remainings elements.
Blazor App handles only an anchor by.
After clicking on an anchor, predefined methods in blazor.js try to invoke navigation handler in Blazor App using a Mono WebAssembly gateway.
A user can modify this handler, but a specialized component Router implements a default behavior.
The Router finds out all components, which implements an IComponent interface, by a reflection and tries to render the page according to path matching RouteAttribute of a component.
The navigation can be redirected to the server if there is no match.
The second type is events invoked by UI like onchange. These event's callbacks call right CSharp callbacks thanks to RenderTreeBuilder, connecting CSharp callback with element's event.

\section{Peachpie}

\cite{Peachpie} is a modern compiler based on Roslyn and Phalanger project.
It allows compiling PHP into a .NET assembly, which can be executed alongside standard .NET libraries.
Peachpie introduces several structures representing states, scripts, and variables of PHP written in Csharp.
The first of them is a context representing one request to PHP code.
The context consists of superglobals, global variables, declared functions, declared and included scripts.
The possibility of saving the context and using it later is a significant advantage used in the solution.
The context can also be considered as a configuration of the incoming script's execution.
All information about a request can be arranged to mock every situation on the server-side.
The compiler offers a dedicated type of assembly for PHP libraries.
Using this assembly can add additional functions, which can provide an extra nonstandard functionality as an interaction with a browser.
Another advantage of the compiler is the great interoperability between PHP and .NET.
An option to work with Csharp objects, attributes and calling methods will become crucial for achieving advanced interaction between Blazor and PHP.

However, there are limitations following from differences in the languages and the stage of development.
Availability of PHP extensions depends on binding these functions to CSharp code which gives equivalent results. The time and memory complexity of this code can be tricky in Blazor.
The previously mentioned interoperability has limits as well.
Csharp constructs like structs and asynchronous methods are undefined in PHP.

\section{PHP}

This section describes the language with standard practice for website development.
PHP was designed for generating a page on the server-side.
This purpose creates several features that make website development easy.
PHP has dynamic types in order to concentrate more on intuitive usage.
The most of PHP application has the semantic of one-way pass.
A request income to the server, which calls the script.
The script generates a response and shutdowns.
The intention of being a server-side language introduces unique global variables representing an incoming request.
These variables are dictionaries.
Three variables are relevant to the thesis.
The GET variable stores parsed query part of the URL.
The POST variable stores variables which are sent by post method.
The FILES variable contains uploaded files.
There is also a SESSION variable, which holds a user session.
This variable will become needless because of the final solution.
Global functions are the most notable characteristic of PHP despite wide-spread object-oriented programming.
It is untypical to use asynchronous methods in PHP.

It is hard to follow web application trends due to fast-changing technologies, but the following concepts are popular for a few years.
The basic pattern is Front Controller.
Usually, the main script invokes other parts of the program, based on the request, to deal with it and send the response back.
An HTML interleaving has appeared to be a helpful method for data binding.
The feature allows inserting a PHP code between HTML.
These fragments do not have to form individual independent blocks of code closed in curly brackets.
The only way how to add user's interaction with the page was web forms before Javascript.
So most developers should be familiar with this concept.

Files uploading consists of two steps.
The first step is save the file's information to object representing file. 
The file is saved as temporary file and the content can be get by standard reading operations.

\chapter{Problem analysis}

We will divide the analysis into two sections.
The first section relates to defining requirements, which the proposed solution will solve.
Four scenarios will describe these requirements, and they will point to the resulting benefits achieved by the scenarios.
The second section will observe available architectures in view of used technologies, and it will describe the solution's architecture.

\section{Scenarios}
We will introduce potential scenarios which will aim to use the integration in different ways.
These scenarios will be essential for our proposed solution.
We will analyze them, and we will suggest ways how to solve them with mentioned technologies.
We will point to related limitations of the ways.
\par
The first scenario moves a website written in PHP to a client-side.
It can help save the server's resources by loading the website's major to a client using one request.
We can imagine a standard PHP website using a Front controller pattern.
We want to handle all navigation by the Main script, distributing an additional workload to other scripts.
The URL information should be accessible in the super global \$\_GET alongside the query.
Scripts should render a page by the interleaving or echo.
Rendering should be triggered once per navigation, which means clicking at an anchor tag.
We should choose a Context duration If we want to use the same context for a whole component life or change it after navigation.
This extension brings a new look at PHP programing when we can utilize saving the context among navigations.
A problem comes with external resources like images.
These resources are needed, while a particular part of the page references them.
This complication has to make another request to the server for obtaining the resource.
\par
The second scenario aims to inject a PHP code to the Blazor page.
The page will do some data processing.
There exist a dedicated PHP library solving this processing, and we are familiar with it.
We should write a PHP script using the library and inject it as a part of the page.
The PHP script should interact with a client by a Form tag to avoid Javascript and advanced interaction with Blazor.
Get and post methods should be enabled and should use the correct superglobals.
There should be file support that will enable loading and saving files from the script.
A PHP script should be rendered as in the previous example.
\par
The third scenario aims to fully utilize aspects of Blazor and move it into the PHP world.
We should be able to inherit from C\# ComponentBase and call C\# Blazor interfaces from PHP.
The scenario is intended for users, which have a notion about Blazor functionality and want to make faster the rendering time for demanding web applications.
The solution should offer to help constructs to improve interaction with Blazor in PHP.
In the end, we should be able to place the web application into the desired place in the Blazor application.
\par
The fourth scenario combines previous scenarios.
We should be able to add a PHP website as a part of the Blazor website.
The PHP website should be able to navigate the component created in the third scenario.

\section{Architecture analysis}
\change[inline]{From now on, everthing are just parts of previous text.}










This chapter describes the main problems together with using technologies to help to solve them.
In the end, we will introduce the proposed solution to the problem.

Current problems of migrating PHP to a browser divides into different types of tasks that have to be done.
In the beginning, we have to load scripts alongside the Blazor app into a browser. 
Afterward, navigation to the scripts is essential for making the PHP website client-side.
The ability to find the desired script should be adopted to Blazor's environment in order to combine PHP scripts with Razor pages.
Mocking the server role on the client-side will be another important feature to make this migration familiar with standard PHP usage.
It consists of managing superglobals like GET or POST, file management.
Because we can save the application state, we bring a new feature of choosing the preservation of the script's context to the next evaluation.
The interaction with a user is a challenge for PHP due to its server role.
And the last problem will be with evaluating the PHP code.
Rendering the whole page is a demanding computation.
It can be critical for sections, which tend to change their content often.
 
\subsection{Proposed solution}
The main idea of migrating PHP to a browser is to integrate Peachpie with Blazor.
In the beginning, we have to think of how we can put the scripts into a browser.
Peachpie can solve it.
It compiles our scripts into .NET assembly, which consists of all information about the scripts.
We can reference it from a Blazor app as a standard CSharp code when we have the assembly.
We have to ensure that compiler will think the application uses the assembly due to cutting unnecessarily assemblies mentioned in the Blazor section.
The process results in loading the assembly alongside the Blazor application into a browser.
However, this could be considered as the major part of the problem.
How to reference and evaluate the scripts is not clear, and there will be many decisions that will not be silver bullets.

We can use a Blazor component class in order to represent a particular script in the Blazor application.
This component should be able to find and evaluate a specified script from the assembly.
This approach can benefit from the component's reusability.
Afterward, we will be able to compose the component with others to make the desired layout.
Before that, we have to make finding and evaluation the script clear.

From this time, we have to distinguish between the purposes of the PHP code.
We will create two components for each of them.
We make the reason clear later in the section and describe detail in the following chapter.

The first purpose is to free the script from Blazor.
It is done by finding the script by the name obtained from a component's parameter or the URL.
The script's evaluation is done via caching its output and adding it as markup text to the RenderTreeBuilder.
It is is a good approach for transparent use of Blazor, but it is ineffective for often rendering.

\change[inline]{User interaction -> forms}
\change[inline]{Superglobals get, post, files}

The second purpose is to offer the complete interface of Blazor.
An inheritance can achieve this.
The PHP class can inherit the CSharp component class due to Peachpie.
The consequence of this is accessible Blazor functions for rendering the content.

However, these purposes are different. They can be combined due to a Component interface.

\change[inline]{Full control over the rendering}







\chapter{Solution}

This chapter describes the complete solution, solving the use cases.
We start with an overview of the solution parts.
Then, we give a detailed description of each part.
\par
The solution consists of four projects, which will form the resulting Blazor website containing PHP scripts.
In Figure \ref{img13:infrastructure}, we can see these projects as green rectangles.
The \textit{Server} project references \textit{Blazor App}, containg a part of the website, and \textit{Peachpie.Blazor}, containing an additional support code.
The server cares about serving the Blazor website and its Static Web Assets.
The next project is our library containing an API for including PHP scripts to the website.
There are \texttt{PhpComponent} and \texttt{PhpScriptProvider}, mentioned earlier, together with additional code support necessary for the correct functionality.
There is the Blazor App project, which becomes the environment for running PHP scripts in a browser.
The project references \textit{PHP scripts} and \textit{Peachpie.Blazor}, which content is used to maintain PHP scripts.
We can see the user's defined scripts as .NET project compiled by Peachpie in \textit{PHP scripts}.
\textit{Blazor App} injects the scripts using the components.
\par
The first section aims at \texttt{PhpComponent}.
It introduces the implementation problems connected to creating render demanding applications and solves them.
The second section talks about \texttt{PhpScriptProvider}.
It suggests a convenient way how to include the scripts into a browser, and it presents the component design.
The last section aims at the server settings.
\par
\begin{figure}[!b]\centering
\includegraphics[scale=0.9]{./img/SolutionInfrastructure}
\caption{The solution infrastructure. Green rectangles represent projects. Arrows represent a references.}
\label{img13:infrastructure}
\end{figure} 

\section{PhpComponent}

In the beginning, we introduce problems, which are related to the \textit{PhpComponent} use case.
Then, we suggest a solution and design \texttt{PhpComponent} class.
\par
The first problem causes PHP, which does not know structs and method overloading.
Structs are necessary to work with \texttt{RenderTreeBuilder}, which contains API for adding callbacks handling element events, as we can see in Figure \ref{img14:callback}.
This API uses method overloading in many methods.
\texttt{AddAttribute} is an example where we can write various types of the attribute value.
One of the values can be \texttt{EventCallback} struct representing an event handler.
The struct contains static property \texttt{Factory}, which is a class containing methods for creating callbacks.
\par
\begin{figure}
\begin{lstlisting}
__builder.OpenElement(5, "button");
__builder.AddAttribute(7, "onclick", 
		EventCallback.Factory.Create<MouseEventArgs>((object)this, 
				(Action)IncrementCount));
__builder.AddContent(8, "Click me");
__builder.CloseElement();
\end{lstlisting}
\caption{Fragment of code adding a button element with an event handler.}
\label{img14:callback}
\end{figure}
\par
Peachpie enables using structs in PHP code. 
However, there are limitations at the time of writing, which force us to make workarounds.
We try to rewrite the previous example in PHP code using Peachpie.
We create a component, which inherits {textttComponentBase}. 
Afterward, we override the method for building a render tree and implements the body.
The fragment of the body can be seen in Figure \ref{img15:problems}, where we try using workarounds to make the example functional.
There is the first issue in line 5, where Peachpie does not allow us to access a static property of struct.
It results in a runtime error, which can be solved by \texttt{Helper} written as a C\# class containing a method, which returns the property.
The second issue causes method overloading when Peachpie can not choose the correct version of the \texttt{Create} method in line 8.
Peachpie defines a type of PHP function, \texttt{IPhpCallable}, which can cause the issue
However, if we wrap this function into the correct type by a helper function, the problem remains.
The workaround can be another helper method, which will have a different name for each overload of this method.
However, we come to a compilation error when we want to get an instance of \texttt{EventCallback}.
As we can see in the figure, we tried to use many workarounds, but it is impossible to use some Blazor structures directly in PHP code.
\par
\begin{figure}
\begin{lstlisting}[numbers=left]
use \Microsoft\AspNetCore\Components;
...
$builder->OpenElement(1, "button");
		
//$factory = Components\EventCallback::Factory;
$factory = \ClassLibrary2\Helper::GetFactory();

//$action = function() {\System\Console::WriteLine("Click");}; 
$action = \ClassLibrary2\Helper::GetAction(function() {
	\System\Console::WriteLine("Click");});
		
//$callback = $factory->
//	Create<Components\Web\MouseEventArgs>($this, $action);
$callback = \ClassLibrary2\Helper::
	GetCallback<Components\Web\MouseEventArgs>($factory, 
		$this, $action);

$builder->AddAttribute(2, "onclick", $callback);
$builder->AddContent(3, "Click me");
$builder->CloseElement();
\end{lstlisting}
\caption{Problem of using structs and method overloading. Helper is a class defining workarounds.}
\label{img15:problems}
\end{figure}
\par
To make the example functional, we can hide the struct from PHP code by implementing a C\# helper method using the struct.
The method should have only parameters compatible with PHP types. 
The overloading can be replaced by a different method name for each overload.
Afterward, Peachpie allows us to call the methods from PHP code.
We can use this approach in the \texttt{AddAtribute} method. 
Defining a new method for each overload is a reasonable approach due to a small number of overloads.
Although defining global methods are not the best way, how to do it.
It is a pity that the builder is sealed. 
Although, we can create a wrapper containing the builder and defining method for each overload, which calls the original method in C\# code.
This decision leads us to make a new \texttt{RenderTreeBuilder}  in a different namespace as a wrapper of the original builder.
\par
The next issue relates to rendering time.
\texttt{RenderTreeBuilder} provides a method for adding arbitrary markup text.
The text can contain \texttt{<script>}, but its content is not executed.
At first glance, one can see the method as a convenient way to render the whole content, avoiding using other dedicated methods for building the tree.
These methods accept a sequence number used by the diff algorithm. 
Although using the one method for rendering, the whole component causes slow rendering, which is critical in some applications like games.
The diff algorithm relies on marking the blocks of markup by sequence numbers for optimization in page updates.
When we have only one big block, the diff algorithm can not do anything better than generate an update, which renders the whole page. 
This issue can be seen in the Benchmark section, where we compare the difference between using the one method and utilizing all methods.
Because the builder usage can be complex, we introduce a library for representing tags, helping implement the code using the builder for rendering.
We present library class diagram in Figure \ref{img16:diagram}.
The main idea is to implement the \texttt{iBlazorWritable} interface, which writes the class content into the builder.
An example of a class is \texttt{Tag}, which represents an arbitrary tag.
Because a tag can contain other tags using sequence numbers, we have to keep the currently used sequence number used in the diff algorithm.
For this purpose, the \texttt{writeTreeBuilder} method gets the actual sequence number and returns the last unused number.
This API should hide separated class logics for rendering.
We offer the basic implementation of this method, which renders the content with a dynamic sequence numbering. 
However, a programmer can override the method because sequence numbering is impossible to predefine in advance to make the most effective updates.
Another abstraction is \texttt{AttributeCollerction}, which offers convenient interface for working with attributes by implementing PHP \texttt{ArrayAccess}.
\par
\begin{figure}\centering
\includegraphics[scale=0.8]{./img/ComponentLibrary}
\caption{Class diagram of supporting library for writting tags.}
\label{img16:diagram}
\end{figure} 
\par
The next barrier is assigning handlers to C\# events in PHP code.
Peachpie does not either support accessing the events.
Thus, we can not directly use a class like \texttt{Timer}, which is useful in \textit{PhpComponent} use case for updating the screen every period.
The issue can be solved by helper methods defined in C\# accepting the object, handler, and event name.
Afterward, we can use reflection for obtaining the desired event by name from the object and then assign the \texttt{IPhpCallable} handler to it.
Because \texttt{Timer} is a common object, we create an additional PHP wrapper class, which uses the timer.
Then a programmer avoids to use the workaround defined above.
\par
\begin{figure}\centering
\includegraphics[scale=0.8]{./img/PhpComponentSolution}
\caption{Class diagram of the use case solution.}
\label{img17:solution}
\end{figure}
\par
When we already presented the problems, we can introduce the architecture of \texttt{PhpComponent} and solution of \textit{PhpComponent} use case.
The component hides the original function for rendering and replaces it with our version of the builder, as shown in Figure \ref{img17:solution}.
It results in transparent usage of the builder in the inherited class.
The builder is just a wrapper, so the programmer can use the original builder by accessing its property.
Additional, there is a library for creating tags, which should make the builder usage easier.
For assigning PHP handlers to C\# events, there is a universal helper.
Furthermore, the last feature is a timer wrapper, which uses the C\# timer, offering a convenient API.

\section{PhpScriptProvider}

In beginning of this section we introduce the component main parts, which gives an overview of the component composition.
We divide component duties like navigation or script execution into subsections
becuase the component consists of many processes, which are complex to describe at once in the structure.
The component functionality should be explained from these sections.
\par
\begin{figure}[b]\centering
\includegraphics[scale=0.8]{./img/PhpScriptProvider}
\caption{Diagram illustrating usage of PhpScriptProvider main parts.}
\label{img18:provider}
\end{figure}
\par
We start with Figure \ref{img18:provider} describing the connections between the main parts.
\texttt{PhpScriptProvider} is a class, representing a Blazor component.
The component manages the following features.
It handles the navigation.
It finds the script by name based on provider mode.
It creates and keeps a PHP context, which is used for script execution.
It executes the script.
These duties contain several steps.
Some of them are maintained by the parts.
As we can see, there is \texttt{PhpComponentRouteManager}, which finds the components, inheriting \texttt{PhpComponent}, based on \texttt{RouteAttribute}.
It enables to navigate Blazor components defined in PHP scripts.
The next part is \texttt{BlazorContext}, which is Peachpie \texttt{Context} designed for Blazor enviroment.
The context initializes superglobals based on URL and submitted forms, manages files uploaded by a form, and redirects the script output to the render tree provided by \texttt{BlazorWriter}, which writes all output to the tree.
In the end, \texttt{FileManager} reads submitted files, downloads them, or deleting them from Browser memory.
\par
We give a precious description of \texttt{PhpScriptProvider} in order to prepare a context for explaining the feature functionalities.
The provider consists of many properties.
Some of them are injected by the dispatcher like \texttt{NavigationManager} or \texttt{IJSRuntime}, which is a service providing interoperability with Javascript.
Others can be parametrized, like \texttt{Type} determining the mode of provider, \texttt{ContextLifetime} determining the persistency of script context, or \texttt{ScriptName} determining the executing script when the mode \textit{Script} is set.
These properties influence the component method behaviors.
The first method is \texttt{Attach}, which assigns a render handle providing \texttt{RenderTreeBuilder}, and registers a navigation handler creating the context and calling \texttt{Refresh}.
The second method,\texttt{SetParameters}, cares about calling \texttt{Refresh}, and creating the context as well.
\texttt{Refresh} finds a script or a component based on properties, assigns the superglobals, and calls \texttt{Render} which renders the component or executes the script.
\texttt{OnAfterRender} cares about enabling the forms to send data back to Blazor.
Some of these methods are called by Blazor framework providing the component lifecycle.

\subsection{Navigation}
\subsection{Script Execution}
We start with \texttt{BlazorWriter}, which inherits \texttt{TextWriter}.
The inheritance allows to use the writer as \texttt{BlazorContext} output writer, which manipulates with script output.
The writer consists of a buffer and \texttt{RenderTreeBuilder}.
The main usage is to write any string to the writer, which adds it to the buffer.
In the end, the writer flushes it into the builder by \texttt{AddMarkUpContent}.
It results in threating the whole script output as one modification by diff algorithm, which causes the whole page update instead of smaller necessary updates.
This disadvantage relates to following limitations.
\texttt{AddMarkUpContent} does not allow to add incomplete markup text, meaning that tags are not properly closed.
Thus, we can not divide the text into smaller parts, because of HTML nature, where tags are coupled by other tags.
The second possibility is to parse the output to differ the types of HTML entities, which is no suitable, because of parser complexity.
It is important to dispose the writer after the rendering, because the same builder can not be repeatedly used.
\subsection{Forms}

\par

\section{Server}
\change[inline]{serving web static assets}
\chapter{Examples}

This chapter demonstrates the usage of our solution by four examples, which are inspired by the use cases mentioned earlier.
We describe example structure, show important blocks of code, which has to be added to project.
At the end, we make snapshots of the websites.

\section{Web}

Web example is inspired by the first use case, which moves the website on a client side.
It consists of four projects. \texttt{Blazor.Client} represents the client application containing \texttt{Program.cs}, which sets \texttt{WebAssemblyBuilder} with a root component, \texttt{PhpScriptProvider} as we can see in Figure \ref{img20:program}.
The project references our library and the project containing PHP scripts.
The provider has default settings, which are the \textit{Router} type and the \textit{OnNavigationChanged} mode of \texttt{BlazorContext}.
Additionally, we have to link the Javascript script from our library to \texttt{index.html} in order to use it during the runtime.
\begin{figure}[H]
\begin{lstlisting}
public static async Task Main(string[] args)
{
	var builder = WebAssemblyHostBuilder.CreateDefault(args);

	// Configure logging
	builder.Logging.SetMinimumLevel(LogLevel.Debug);

	// Add PHP
	builder.AddPhp(new[] { typeof(force).Assembly });
	builder.RootComponents.Add(typeof(PhpScriptProvider), "#app");
            
	await builder.Build().RunAsync();
}
\end{lstlisting}
\caption{The Main method in Program.cs}
\label{img20:program}
\end{figure}
\par
\texttt{Blazor.Server} provides the application with other static resources to the client.
It references \texttt{Blazor.Client} and our library, which provides an embedded file representing the Javascript script.
We have to add middlewares for serving the resources in the last two projects, as we can see in Figure \ref{img21:server}.
\par
\begin{figure}[H]
\begin{lstlisting}
var fileProvider = new ManifestEmbeddedFileProvider(
	typeof(PhpBlazor.BlazorContext).Assembly);
app.UseStaticFiles(new StaticFileOptions() {
	 FileProvider = fileProvider });

app.UseStaticFiles(new StaticFileOptions
{
	FileProvider = new PhysicalFileProvider(
		Path to static resources, "Web\\wwwroot")),
	RequestPath = "/Web"
});
\end{lstlisting}
\caption{A part of the Configure method contained in the StartUp class, which is defined in Startup.cs}
\label{img21:server}
\end{figure}
\par
Then, there is our library \texttt{Peachpie.Blazor} providing API for executing PHP scripts.
The last project contains the programmer's defined PHP scripts forming the web of some software company.
The project uses Peachpie \ac{SDK} for compiling the scripts.
We can see the project content in Figure \ref{img22:web}.
The website has a simple layout defined in \texttt{defaultLayout.php} referencing pages about the founder, the company, and the community.
The \texttt{me.php} page contains an image, \texttt{logo.png}, which is loaded by common tag \texttt{<img alt="Logo" src="Web/images/logo.png"/>}.
We can see the \texttt{force.php} script containing empty \texttt{force} class, which is used in \texttt{Blazor.Client} to force loading of the \texttt{Web} assembly into \texttt{Blazor.Client}.
\par
\begin{figure}[H]\centering
\includegraphics[scale=0.9]{./img/WebStructure}
\caption{The Web solution structure.}
\label{img22:web}
\end{figure} 
\par
An interesting page is \texttt{index.php}, which is default page of the website, shown in Figure \ref{img23:index}.
It uses script inclusion to add the head section.
Then, there is a Javascript call, which uses our predefined API, which causes showing the alert with the message when the page loads.
The whole page uses HTML interleaving.
Static resources like images can be referenced by ordinary tags.
There is also an anchor to the component defined in \texttt{SimpleComponent.php}, which can be seen in Figure \ref{img24:component}.
This page is transparently rendered by our \texttt{PhpScriptProvider}, which evaluates the whole script and adds the output as a markup text to the builder.
We can try persistent context by changing the mode.
Then, we can assign some global variable, \texttt{\$msg}, as we can see in Figure \ref{img23:index}.
When we navigate to the component, the variable remains, and we can see the content of the variable defined on the previous page.
\par
\begin{figure}[H]
\begin{lstlisting}
<?php
    require("/headers/defaultHeader.php");
    CallJsVoid("window.alert", "Hello from PHP script.");
?>
<h1>Index</h1>
...
<img src="Web/images/offlineMode.png" width="600" height="200"/>
...
Try navigating to the <a href="/simpleComponent">component</a>

<?php
    $msg="Hello from provider";
    require("/footers/defaultFooter.php");
?>
\end{lstlisting}
\caption{index.php}
\label{img23:index}
\end{figure}
\par
The \texttt{SimpleComponent} component extends our \texttt{PhpComponent} and uses \texttt{RouteAttribute} for routing.
The possibility of using attributes was added in the PHP8.0 version, and Peachpie supports them.
We can see implementing the \texttt{BuilderRenderTree} method, which uses our builder to add markup content to the page.
A more complex example of using the component we can see in \textit{PhpComponent} example.
\par
\begin{figure}
\begin{lstlisting}
<?php namespace something;
use \Microsoft\AspNetCore\Components as Microsoft;

#[Microsoft\RouteAttribute("simpleComponent")]
class ExampleComponent extends \PhpBlazor\PhpComponent {	

public function BuildRenderTree($builder) : void {
	global $msg;
	$builder->AddMarkupContent(0, "<h1>Simple component</h1>");
	$builder->AddMarkupContent(1, "<p>msg = " . $msg . "</p>");
}

}
\end{lstlisting}
\caption{SimpleComponent.php}
\label{img24:component}
\end{figure}
\par
We can see using \texttt{\$\_GET} superglobal in Figure \ref{img25:developer}, where we decide to show its content based on the URL query.
When we have the \textit{OnNavigationChanged} mode for the context, and we refresh the page after navigation to a developer, then we can see the anchors to developers.
It is caused by creating a new context between navigation, so the variables are disposed.
With the second mode, we still have the info page of the firstly navigated developer because the variables in the context remain.
\par
\begin{figure}[H]
\begin{lstlisting}
<?php
    require("/headers/defaultHeader.php");
?>
<?php
if (isset($_GET["developer"])) { 
    $name = $_GET["developer"];
    require("/community/developer$name.php");
} else {
?>
...
<p>Get more info about 
<a href="/community/developers.php?developer=Richard">Richard</a>.
</p>
...
<?php } ?>
<?php
    require("/footers/defaultFooter.php");
?>
\end{lstlisting}
\caption{developers.php}
\label{img25:developer}
\end{figure}

\section{OneScript}

In this example, we aim at the second use case.
The solution contains four project again.
\texttt{Blazor.Server} and \texttt{Peachpie.Blazor} are same.
\texttt{Blazor.Client} contains now common Razor pages, and has \texttt{Router} as a root component.
We create several scripts in the \texttt{Scripts} project to enrich the website with content generated from the PHP scripts.
The website contains three Razor pages: \texttt{Index.razor}, \texttt{PhpGateway}, and \texttt{PhpScript}.
The first page uses \texttt{PhpScriptProvider} to navigate \texttt{index.php}.
Using the provider is straightforward.
\par
We want to show the calling PHP function from Javascript in Figure \ref{img26:index}.
As we can see, it is effortless to call it.
The \texttt{callPHP} function accepts the function name and object to serialize as an function parameter.
When the script is rendered, the context contains defined \texttt{CallPHP} function.
We click on the button, which invokes \texttt{Call} method on the context, which invokes the desired function.
Then, we deserialize the parameter.
There is an interesting thing about using \texttt{echo}, \texttt{print}, etc. when the script is not rendered.
The context provides the second writer, which uses \texttt{Console} as the output.
It causes printing the message into the web browser console.
\begin{figure}
\begin{lstlisting}
...
<p>Click and look at console output</p>
<button onclick="window.php.callPHP('CallPHP',
	{ name : 'Bon', surname: 'Jovi'});">PHP</button>
<?php

function CallPHP($data)
{
    $json = json_decode($data); 

	echo "Hello " . $json->name . " ";
	echo $json->surname .  " from PHP\n";
}
\end{lstlisting}
\caption{index.php}
\label{img26:index}
\end{figure}
\par
Another part of the website uses forms to demonstrates \texttt{GET} and \texttt{POST} method.
We can see it in \texttt{php} folder, where are three examples of forms using both methods and file loading.
These examples can be navigated based on their names due to the unspecified URL of the Razor page, which uses the provider.
After navigation to this page, the provider gets the script name from the URL.
\par
\begin{figure}[!b]\centering
\includegraphics[scale=0.4]{./img/graph}
\caption{Application for visualising a graph.}
\label{img27:graph}
\end{figure} 
\par
The last part aims at the persistent context and using forms as interaction with the user.
It is a common approach in PHP, and we can use it on a client side due to our solution.
There is a simple application enabling us to visualize a graph, as we can see in the folder \texttt{fileManagment}.
The application can upload a CSV file containing a graph or generate a new one based on the given parameters.
It then visualizes the graph and enables it to be downloaded as a CSV file, which can be used for the next visualization, as shown in Figure \ref{img27:graph}.
We use PHP library for parsing the file, which demonstrates a possibility to utilize the already created library on a client side.
The application has the main script,\texttt{fileManagment/index.php}, which recognizes what to do based on superglobals and saved variables.
It is possible due to context persistence.

\section{PhpComponent}
\section{AllTogether}
\chapter{Benchmarks}
\change[inline]{Rendering speed(Asteroids first version vs. the current)}
\change[inline]{Problem with gd library}

\chapter*{Conclusion}
\addcontentsline{toc}{chapter}{Conclusion}

\section*{Future work}
\addcontentsline{toc}{section}{Future work}

%%% Bibliography
\include{bibliography}

%%% Figures used in the thesis (consider if this is needed)
\listoffigures

%%% Tables used in the thesis (consider if this is needed)
%%% In mathematical theses, it could be better to move the list of tables to the beginning of the thesis.
\listoftables

%%% Abbreviations used in the thesis, if any, including their explanation
%%% In mathematical theses, it could be better to move the list of abbreviations to the beginning of the thesis.
\chapwithtoc{List of Abbreviations}

\begin{acronym}
 \acro{HTTP}{Hypertext Transfer Protocol}
 \acro{HTML}{HyperText Markup Language}
 \acro{CSS}{Cascading Style Sheets}
 \acro{WASM}{WebAssembly}
 \acro{W3C}{World Wide Web Consortium}
 \acro{URL}{Uniform Resource Locator}
 \acro{DOM}{Document Object Model}
 \acro{ES}{ECMAScript}
 \acro{SDK}{Software Development Tools}
 \acro{FPS}{Frames Per Second}
\end{acronym}

%%% Attachments to the bachelor thesis, if any. Each attachment must be
%%% referred to at least once from the text of the thesis. Attachments
%%% are numbered.
%%%
%%% The printed version should preferably contain attachments, which can be
%%% read (additional tables and charts, supplementary text, examples of
%%% program output, etc.). The electronic version is more suited for attachments
%%% which will likely be used in an electronic form rather than read (program
%%% source code, data files, interactive charts, etc.). Electronic attachments
%%% should be uploaded to SIS and optionally also included in the thesis on a~CD/DVD.
%%% Allowed file formats are specified in provision of the rector no. 72/2017.
\appendix
\chapter{Attachments}

\section{First Attachment}

\openright
\end{document}

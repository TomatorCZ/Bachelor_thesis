\chapter{Existing technologies}

\section{WebAssembly}

\cite{WebAssembly} is a new code format which can be run in today's browsers. 
It has compact byte format and it's performance is near to native code. 
WebAssembly is designed to be a compiling target of popular low-level languages like C or C++ due to it's memory model. 
It should be able to support languages with carbage collector in the future. Advantage of this format is similarity with Javascript module ES2015 after compilation into machine code. 
This enables browsers to execute it by Javascript runtime. 
So it's security is same as code written in Javascript. 
Because of the same runtime WebAssembly can call Javascript and vice versa.

Despite of supporting to run WebAssembly in browser, the browser can't load it as a normal ES2015 module yet.
WebAssembly Javasrcipt API was created in order to be able to load a WebAssembly to browser using JavaScript.

\section{Mono}

Mono is a .NET runtime which aims to mobile platforms. 
Recently, they started to support \cite{compilation} into WebAssembly.
This support allows executing CIL inside browsers.
The compilation has two modes. 
One of the modes compiles only Mono runtime which then  can executes .dll files without further compilation of them into WebAssembly.
A consequence of this compilation is enabling to call Javascript and WebAPI from CIL.

\section{Blazor}

Blazor is a framework which provides a convinient way how to write dynamic web pages in CSharp.
It offers two \cite{Hosting models} how to build your websites.

The first one consists of two parts. Server part cares about serving web pages to client via SignalR.
The thesis uses the second way which Microsoft refers as Blazor WebAssembly.

Blazor WebAssembly uses server and client part as well.
The main difference is in where the webpage is put together.
The client side cares about rendering the page.
This is possible due to Mono and WebAssembly Javascript interop which enables to modify DOM from CSharp using Mono Webassembly.
For this purpose, there are constructs which can configure the behavoir of the web application.
The behavoir is meant as which pages to show or services to offer.
This all settings is done by client project.
The server is used for serving resources of the wep application.
Recources are static assets, .dll libraries including web applivation, mono runtime and so on.

Content of the web application is composed from components.
Component is a class which stands for generating a part of content.
Balzor presents own virtual DOM to reduce changing DOM directly in browser in order to its demending performance.

\change[inline]{TODO: Diff algorithm of Blazor}

Because interleaving of HTML with other language turns out to be useful, the Razor language was introduced.
This Razor differs from Razor used in .cshtml files.
It is adapted to Blazor WebAssebmly enviroment which provides additional features and settings of this application.
Antoher reason for using this format is to free developer from difficult using of mechanism for rendering a page content.

The process of bootstrapping Blazor app to browser folows these steps. 
Server gets a request for Blazor app. 
The server responses with html page, which contains references to Javascript responsible to load the app. The Javascript code fetches remaing resources like .dll libraries and runs Mono module. 
The runtime initialise the application using user defined .dll libraries.

Remaing interaction is maintained by event handling.
I divides it into two type of actions.
\change[inline]{TODO: Navigation}
\change[inline]{TODO: Components Events}

\section{PeachPie}

\change[inline]{TODO: What is PeachPie}
\change[inline]{TODO: Compiling PHP to .NET}

\section{PHP}

\change[inline]{TODO: Classic web page in php}

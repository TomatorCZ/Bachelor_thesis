\chapter*{Introduction}
\addcontentsline{toc}{chapter}{Introduction}

Nowadays, web applications are mostly dynamic in order to offer a convenient user interface.
Classical static pages depend on a markup language HTML, which describes a text's structure in a browser and a rendering algorithm interpreting this language.
The need to adjust content by different styles initiates standardizing CSS language, which enriches pages with a wide graphical content.
However, this combination of CSS and HTML is sometimes sufficient. An opportunity to have full control over a page is necessary to make the page looking similar to a desktop application.
This type of application needs dedicated technology, which allows manipulation with a page structure, reacts to on-page events and controls the browser's behavior.
However, many attempts to create rich Internet applications using technologies like obsoleted Silverlight, developed by Microsoft or Adobe Flash Player, were functional. 
The major disadvantage was the need to add a plugin to a browser.
This reality causes problems with an installation, versions, and other complications connected with plugins.
The solution came with a scripting language, Javascript, which became a standard in most browsers like Google Chrome, Safari, Opera, and Mozzila.
Despite a first looking ineffective, a community devotes a huge amount of resources to make the language optimal for complex tasks.
However, Javascript is a powerful language, there are language-specific features, which are harder replaceable by the language.
There appeared a portable binary-code format for executing programs, WebAssembly (abbreviated WASM) \cite{1} in 2015.
WebAssembly aims to high-performance applications on web pages.
The advantage of WebAssembly is a target of compilation-friendly low-level languages like C or C++ due to its memory model.
Although there are exists ways, how to compile other high-level languages like CSharp or PHP.
Since December 2019, when the W3 Consortium has begun recommending WebAssembly, it is easy to migrate standard desktop languages to modern browsers supporting this recommendation.

The most popular language becomes PHP, talking about a server-side world.
A community of PHP developers is significant.
Thus, there are many PHP libraries for working with client's data, cooperation with databases, and other server tasks.
The possibility two migrate the language PHP together with its conventions to a browser will impact developing dynamic web applications due to the PHP community and the libraries.

An idea of migration PHP to a browser is achievable by a compilation to the previously mentioned WebAssembly.
The project PHP in browser \cite{2} enables running PHP script inside your browser using predefined Javascript API or standard tag for script in HTML.
Although this project enables using PHP on the client-side, there is necessary to have additional knowledge about Javascript to use it in common scenarios.
One of the thesis's goals is to remove this shortage.

The consequence of WASM is an introduction of Blazor \cite{3}.
Blazor is a part of the ASP.NET platform developed by Microsoft.
It provides a runtime and templates for creating dynamic web pages using CSharp with no or at least minimal Javascript support.

Another interesting technology connecting .NET and PHP world is Peachpie \cite{4}.
Peachpie is a modern compiler enabling a transformation of PHP scripts to .NET assembly, which results in running PHP code in .NET runtime.

The last two technologies' connection can be an interesting integration enabling to use PHP in a browser.
The thought of making the integration successful is to chain the compilation of PHP scripts to .NET and using the existing .NET platform to executes the compilated scripts in a browser.
The idea has the potential of joining PHP and CSharp developers to collaborate with their programming languages using a minimum knowledge of the integration.
Another interesting functionality of this approach is a full CSharp, PHP, and JavaScript interop which offers more options for developers and future extensions.

This thesis aims to integrate Peachpie and Blazor to achieve a comfortable way of writing a client-side web application using PHP.
The first chapter addresses the analysis of related work, alongside descriptions of the technologies used in the integration.
The second chapter analyses the problem of running PHP on the client-side and other problems related to used technologies.
The third gives a detailed problem's solution.
There are examples that demonstrate how to use all aspects of the created solution in chapter 4.
In chapter 5, you can see benchmarks that explore the limits of the implementation and compare them with the already existing project.
And the last chapter relates to a conclusion of this solution.

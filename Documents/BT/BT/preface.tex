\chapter*{Introduction}
\addcontentsline{toc}{chapter}{Introduction}

We can divide web applications into two types by roles of server and client.
However, they use different technologies for their purpose.
We will start with common parts.
An internet protocol, HTTP, usually carries out the communication between a server and a client.
A client uses a web browser for requesting a server.
A markup language HTML is essential for describing a page's structure.
A browser is responsible for interpreting and rendering the page's content.
It is a web application's environment for further interaction.
The need to adjust content by different styles initiates standardizing CSS language, which enriches pages with a wide graphical content.
\par
The first type is server-based web applications.
A server prepares the page, makes additional computations related to the request, and sends it back to the client.
Having a business logic on a server-side is the main objective.
The most popular language for server-side scripting becomes PHP.
\change[inline]{Add a statistic for the popularity of PHP}
\par
The second type is client-based web applications, where major business logic is moved to a client's browser.
However, the combination of CSS and HTML is sometimes sufficient.
This type of application needs dedicated technologies, which allow manipulation with a page structure, reacts to on-page events, and controls the browser's behavior.
Many languages were enabling the manipulation, but they were not usually supported by most browsers like Google Chrome, Safari, Opera, and Mozzila.
The scripting language Javascript became a browser standard from these supports.
\par
However, Javascript is a powerful language.
There are language-specific features, which are harder replaceable by the language.
Despite the urge, many technologies like Silverlight, which runs C\# code in a browser, or Adobe Flash Player with Actionscript were deprecated due to insufficient support across the browsers.
There appeared a portable binary-code format for executing programs, WebAssembly (abbreviated WASM) \squarecite{1} in 2015.
WebAssembly aims to secure high-performance applications on web pages.
Interop with Javascript makes the format as powerful as the language.
The advantage of WebAssembly is a being compilation target for many programming languages.
Since December 2019, when the W3 Consortium has begun recommending WebAssembly, it is easy to migrate other languages to the browsers supporting this recommendation.
\par
Many projects use the WASM as a target of compilation.
For example, the project PHP in browser \squarecite{2}.
It enables running PHP script inside our browser using predefined Javascript API or standard tag for HTML script.
Another project is an open-source framework Blazor \squarecite{3} developed by Microsoft.
It provides a runtime, libraries, and interop with Javascript for creating dynamic web pages using C\#.
\par
\change[inline]{Add a statistic for popularity of .NET}
The .NET and PHP popularity led to the creation of the Peachpie compiler.
Peachpie \squarecite{4} tries to make use of the .NET platform and offers PHP compilation to .NET.
It is a modern compiler enabling interop between PHP and C\#.
\par
The project opens the PHP door to Blazor.
An integration between Peachie and Blazor can yield to following benefits.
A community of PHP developers is significant.
Thus, many PHP libraries apply to working with client's data, cooperation with databases, and other server tasks.
The possibility to migrate the language PHP together with its conventions to a browser will impact developing dynamic web applications due to the PHP community and the libraries.
It can join PHP and C\# developers to collaborate with their programming languages using a minimum knowledge of the integration.
Another interesting functionality of this idea is a full C\#, PHP, and JavaScript interop which offers more options for developers and future extensions.
\par
This thesis uses the compilation of PHP scripts to .NET in order to execute PHP in a browser powered by Blazor.
The approach tries to achieve two goals.
The first goal is to enable web development on a client-side with PHP.
There are not libraries supporting this integration.
The second goal is to design the support to offer a convenient way to combine a PHP code with a Blazor.   
\par
The first chapter addresses the analysis of related work, alongside descriptions of the technologies used in the integration.
The second chapter analyses running PHP on the client-side and other problems related to used technologies.
The third gives a detailed problem's solution.
There are examples that demonstrate how to use all aspects of the created solution in chapter 4.
In chapter 5, we can see benchmarks that explore the limits of the implementation and compare them with the already existing project.
And the last chapter relates to a conclusion of this solution.

\chapter*{Conclusion}
\addcontentsline{toc}{chapter}{Conclusion}

We observed the limitations of Peachpie and Blazor.
Then, we considered possible ways, how to combine the PHP script compilation with the Blazor environment to provide executing these scripts on a client side.
The usage of this integration was described by four use cases aiming at different end-users.
The proposed solution results in \texttt{Peachpie.Blazor} library, which offers API for inserting PHP scripts to Razor pages of the Blazor website.
These scripts can utilize different levels of abstraction to cover the problem better.
We are able to work with Blazor structures directly or render the content of PHP scripts due to our solution.
These features help PHP programers to use Blazor and move the script executions to a client side and helps PHP and C\# programers to co-work on a joint website.
\par
We demonstrated the use cases by implementing four examples, which presents different levels of difficulty utilizing the integration.
\par
In the end, we used two benchmarks to reveals issues caused by the used technologies and approaches of using structures for rendering.
We learned that the proper usage of \texttt{RenderTreeBuilder} has the purpose, and C\# library has not to be optimized in the Blazor environment, which can result in lower effectiveness.

\section*{Future work}
\addcontentsline{toc}{section}{Future work}

Peachpie is still developing, and the issues that we met can be solved in the feature versions.
It would be nice to follow these updates and adjust our solution to more utilize the future Peachpie features.
